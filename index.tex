% Options for packages loaded elsewhere
\PassOptionsToPackage{unicode}{hyperref}
\PassOptionsToPackage{hyphens}{url}
%
\documentclass[
]{book}
\usepackage{amsmath,amssymb}
\usepackage{iftex}
\ifPDFTeX
  \usepackage[T1]{fontenc}
  \usepackage[utf8]{inputenc}
  \usepackage{textcomp} % provide euro and other symbols
\else % if luatex or xetex
  \usepackage{unicode-math} % this also loads fontspec
  \defaultfontfeatures{Scale=MatchLowercase}
  \defaultfontfeatures[\rmfamily]{Ligatures=TeX,Scale=1}
\fi
\usepackage{lmodern}
\ifPDFTeX\else
  % xetex/luatex font selection
\fi
% Use upquote if available, for straight quotes in verbatim environments
\IfFileExists{upquote.sty}{\usepackage{upquote}}{}
\IfFileExists{microtype.sty}{% use microtype if available
  \usepackage[]{microtype}
  \UseMicrotypeSet[protrusion]{basicmath} % disable protrusion for tt fonts
}{}
\makeatletter
\@ifundefined{KOMAClassName}{% if non-KOMA class
  \IfFileExists{parskip.sty}{%
    \usepackage{parskip}
  }{% else
    \setlength{\parindent}{0pt}
    \setlength{\parskip}{6pt plus 2pt minus 1pt}}
}{% if KOMA class
  \KOMAoptions{parskip=half}}
\makeatother
\usepackage{xcolor}
\usepackage{longtable,booktabs,array}
\usepackage{calc} % for calculating minipage widths
% Correct order of tables after \paragraph or \subparagraph
\usepackage{etoolbox}
\makeatletter
\patchcmd\longtable{\par}{\if@noskipsec\mbox{}\fi\par}{}{}
\makeatother
% Allow footnotes in longtable head/foot
\IfFileExists{footnotehyper.sty}{\usepackage{footnotehyper}}{\usepackage{footnote}}
\makesavenoteenv{longtable}
\usepackage{graphicx}
\makeatletter
\def\maxwidth{\ifdim\Gin@nat@width>\linewidth\linewidth\else\Gin@nat@width\fi}
\def\maxheight{\ifdim\Gin@nat@height>\textheight\textheight\else\Gin@nat@height\fi}
\makeatother
% Scale images if necessary, so that they will not overflow the page
% margins by default, and it is still possible to overwrite the defaults
% using explicit options in \includegraphics[width, height, ...]{}
\setkeys{Gin}{width=\maxwidth,height=\maxheight,keepaspectratio}
% Set default figure placement to htbp
\makeatletter
\def\fps@figure{htbp}
\makeatother
\setlength{\emergencystretch}{3em} % prevent overfull lines
\providecommand{\tightlist}{%
  \setlength{\itemsep}{0pt}\setlength{\parskip}{0pt}}
\setcounter{secnumdepth}{5}
\usepackage{booktabs}
\usepackage{amssymb}
\usepackage{amsmath}
\usepackage{graphicx}
\usepackage{wasysym}
\usepackage{amsthm}
\usepackage{multirow}
\usepackage{epsf}
\usepackage{tikz}
\usepackage{cancel}
\usepackage{hyperref}
\usepackage{gensymb}
\usepackage{color}
\usepackage{framed}
\setlength{\fboxsep}{.8em}

\newenvironment{blackbox}{
  \definecolor{shadecolor}{rgb}{0, 0, 0}  % black
  \color{white}
  \begin{shaded}}
 {\end{shaded}}
\ifLuaTeX
  \usepackage{selnolig}  % disable illegal ligatures
\fi
\usepackage[]{natbib}
\bibliographystyle{plainnat}
\usepackage{bookmark}
\IfFileExists{xurl.sty}{\usepackage{xurl}}{} % add URL line breaks if available
\urlstyle{same}
\hypersetup{
  pdftitle={Bookdown: Flexible Document Creation in RStudio},
  pdfauthor={Joseph Thiers},
  hidelinks,
  pdfcreator={LaTeX via pandoc}}

\title{Bookdown: Flexible Document Creation in RStudio}
\usepackage{etoolbox}
\makeatletter
\providecommand{\subtitle}[1]{% add subtitle to \maketitle
  \apptocmd{\@title}{\par {\large #1 \par}}{}{}
}
\makeatother
\subtitle{Guide for Students, Researchers, and Professionals}
\author{Joseph Thiers}
\date{}

\begin{document}
\maketitle

{
\setcounter{tocdepth}{1}
\tableofcontents
}
\chapter{Introduction to Bookdown}\label{introduction}

In today's fast-paced academic and professional environments, the ability to create dynamic, reproducible documents is crucial. Bookdown empowers users to combine text, code, and visualizations in a single, streamlined workflow. It is ideal for creating the type of documentation that best fits your needs, whether that be single-page assignments, reports, academic papers, or even full-length books.

\section{Why RStudio and Bookdown?}\label{why-rstudio-and-bookdown}

Bookdown offers a range of benefits:
- Seamlessly integrates text, code, and figures.
- Supports multiple output formats (HTML, PDF, EPUB).
- Simplifies the creation of reproducible and professional documents.
- Ideal for mathematics, statistics, and data science professionals.

\section{What You'll Learn in This Tutorial}\label{what-youll-learn-in-this-tutorial}

This tutorial will guide you through the essential aspects of using Bookdown:

\begin{itemize}
\tightlist
\item
  \textbf{Chapter \ref{introduction}}: Introduction to Bookdown -- Learn about its purpose and benefits for structured documentation.
\item
  \textbf{Chapter \ref{chapter2}}: Getting Started -- Install R, RStudio, and Bookdown, create a project, and render your first book.
\item
  \textbf{Chapter \ref{chapter3}}: Writing Content -- Organize chapters, use Markdown, format text, add code chunks, and images.
\item
  \textbf{Chapter \ref{chapter4}}: Cross-Referencing -- Reference sections, figures, tables, and equations effectively.
\item
  \textbf{Chapter \ref{chapter5}}: LaTeX -- Add equations, theorems, lemmas, and proofs.
\item
  \textbf{Chapter \ref{chapter6}}: Advanced Features -- Manage citations, use LaTeX packages, and more!
\item
  \textbf{Chapter \ref{chapter7}}: Customizing Output -- Configure formats like HTML, PDF, and EPUB, and style your book with CSS or LaTeX.
\item
  \textbf{Chapter \ref{latexdistributions}}: LaTeX Distributions - Different distributions available.
\item
  \textbf{Chapter \ref{markdownadvanced}}: Advanced Text Formatting Options - Advanced Markdown and Pandoc code to stylize the book to your needs.
\item
  \textbf{Chapter \ref{examplepaper}}: Example Document: Union Earnings Analysis
\end{itemize}

By the end of this tutorial, you'll have the knowledge to create, customize, and publish professional-grade documents.

  \bibliography{book.bib}

\end{document}
